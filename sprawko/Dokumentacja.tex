\documentclass{report}
\usepackage{polski} %przydatne podczas składania dokumentów w j. polskim
\usepackage{setspace}
\usepackage[utf8]{inputenc} %kodowanie znaków, zależne od systemu
\usepackage[T1]{fontenc} %poprawne składanie polskich czcionek
\usepackage{subfigure}
\usepackage{psfrag}
\usepackage{tgheros}
\renewcommand{\familydefault}{\sfdefault}
%pakiety dodające dużo dodatkowych poleceń matematycznych
\usepackage{amsmath}
\usepackage{amsfonts}
%pakiety wspomagające i poprawiające składanie tabel
\usepackage{supertabular}
\usepackage{array}
\usepackage{tabularx}
\usepackage{hhline}
\usepackage{float}
\usepackage{indentfirst}
\usepackage{color}
\usepackage{enumerate}
\usepackage[a4paper, total={6in, 10in}]{geometry}
\usepackage{etoolbox}
\makeatletter
\patchcmd{\chapter}{\if@openright\cleardoublepage\else\clearpage\fi}{}{}{}
\makeatother
\usepackage{titlesec} 

%definicje własnych poleceń
\newcommand{\R}{I\!\!R} %symbol liczb rzeczywistych, działa tylko w trybie matematycznym
\newtheorem{theorem}{Twierdzenie}[section] %nowe otoczenie do składania twierdzeń

\newcommand*{\fancychapterstyle}{%
\titleformat{\chapter}{\bfseries\huge}{\filright}{1ex}
{\thechapter . }
}

%tutaj zaczyna się właściwa treść dokumentu
\begin{document}
	\bibliographystyle{unsrt} %tylko gdy używamy BibTeXa, ustawia polski styl bibliografii
	\clearpage
\thispagestyle{empty}

\setstretch{2}
\begin{center}
	\vspace*{2cm}
	\textbf{{\Huge ZASTOSOWANIE INFORMATYKI W~MEDYCYNIE}}
	
	\vspace{0.8cm}	
	\textbf{{\Huge PROJEKT}}
	
	\vspace{1.6cm}	
	\textbf{{\huge TEMAT:\textit{ Detekcja zespołu QRS w sygnale EKG}}}
	
	\vspace{1.2cm}	
	\textbf{{\LARGE PROWADZĄCY:}}
	
	\vspace{0.1cm}	
	\textbf{{\LARGE dr hab. inż. Robert Burduk}}
\end{center}

\begin{flushright}
	\vspace{7cm}
	\textbf{{\LARGE AUTOR:}}
	
	\vspace{0.1cm}	
	\textbf{{\LARGE Radosław Taborski 209347}}
\end{flushright}

	
	\tableofcontents %spis treści
	\newpage
	\fancychapterstyle
	\chapter{Cel projektu}
	Celem projektu jest stworzenia aplikacji, która ma analizować sygnał EKG pod kątem obecności w nim zespołu QRS. Aplikacja wykrywa początki oraz końce zespołów, a także załamki: Q, R oraz S.\newline
	
	\chapter{Algorytm}
	Celem projektu jest stworzenia aplikacji, która ma analizować sygnał EKG pod kątem obecności w nim zespołu QRS. Aplikacja wykrywa początki oraz końce zespołów, a także załamki: Q, R oraz S.\newline
	
	\chapter{Implementacja}
	
	\chapter{Analiza wyników}
	
	\chapter{Bibliografia}

\vspace*{1cm}

\begin{enumerate}[\lbrack 1\rbrack]
	\item http://www.robots.ox.ac.uk/~gari/teaching/cdt/A3/readings/ECG/Pan+Tompkins.pdf \label{bib:algorithm} \newline
	
	\item http://people.ece.cornell.edu/land/courses/ece5030/labs/s2013/QRS\_detect\_review.pdf \label{bib:method}
	
	\item http://physionet.org/tutorials/physiobank-text.shtml \label{bib:physio} \newline
	
\end{enumerate}


\end{document}
